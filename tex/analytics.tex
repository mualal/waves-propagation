\documentclass[main.tex]{subfiles}

\begin{document}
\section{Аналитические выводы (для решёток с одинаковой жёсткостью).}

Ищем решение уравнений движения в следующем виде (метод трёх волн):
\beq
u_{n,m}=
\begin{cases}
	A_{I}e^{\im\left(\Omega t-ak_1^xn-ak_1^ym\right)}+A_{R}e^{\im\left(\Omega t+ak_1^xn-ak_1^ym\right)},\,\,\,\,\,n<0\\
	A_{T}e^{\im\left(\Omega t-ak_2^xn-ak_2^ym\right)},\,\,\,\,\,n\geqslant 0
\end{cases}
\eeq

Дисперсионное соотношение для решётки:
\beq
m_i\Omega^2=4c\left(\sin^2{\frac{k_i^xa}{2}}+\sin^2{\frac{k_i^ya}{2}}\right)
\eeq

По условию волна падает под углом $\gamma$ (угол к нормали интерфейса), поэтому

Уравнения движения для частиц интерфейса:
\beq
\begin{cases}
	m_1\ddot{u}_{-1,m}=c\left(u_{0,m}+u_{-1,m+1}+u_{-2,m}+u_{-1,m-1}-4u_{-1,m}\right)\\
	m_2\ddot{u}_{0,m}=c\left(u_{1,m}+u_{0,m+1}+u_{-1,m}+u_{0,m-1}-4u_{0,m}\right)
\end{cases}
\eeq

\beq
\frac{A_T}{A_I}=\frac{e^{\im k_1^xa}-e^{-\im k_1^xa}}{e^{\im k_2^xa}-e^{-\im k_1^xa}}\cdot\frac{e^{\im k_2^y am}}{e^{\im k_1^y am}}
\eeq

\beq
T=\frac{m_2g_2^x}{m_1g_1^x}\left|\frac{e^{\im k_1^xa}-e^{-\im k_1^xa}}{e^{\im k_2^xa}-e^{-\im k_1^xa}}\right|^2,
\eeq
где
$$
g_i^x=\frac{d\Omega}{dk_i^x}=\frac{ac}{\Omega m_i}\sin{k_i^xa};\,\,\,\,\,
g_i^y=\frac{d\Omega}{dk_i^y}=\frac{ac}{\Omega m_i}\sin{k_i^ya};\,\,\,\,\,
g_i=\sqrt{\left(g_i^x\right)^2+\left(g_i^y\right)^2}
$$

%Из геометрии $k_1^y=k_2^y$.

\section{Сравнение численных результатов с аналитическим решением (для цепочек).}

Аналитическое решение для отношения амплитуд проходящей и падающей волн:
\beq
\frac{A_T}{A_I}=\frac{2\im c_{12}\sin{k_1}}{c_{12}\left(1-e^{-\im k_1}\right)+c_2\left(e^{\im k_2}-1\right)\left(1+e^{-\im k_1}\left(c_{12}-c_1\right)/c_1\right)}
\eeq

Аналитическое решение для коэффициента прохождения:
\beq
T=\frac{m_2g_2\left|A_T\right|^2}{m_1g_1\left|A_I\right|^2}=\frac{m_2g_2}{m_1g_1}\left(\frac{2\im c_{12}\sin{k_1}}{c_{12}\left(1-e^{-\im k_1}\right)+c_2\left(e^{\im k_2}-1\right)\left(1+e^{-\im k_1}\left(c_{12}-c_1\right)/c_1\right)}\right)^2,
\eeq
где
$$
k_1=\frac{2}{a}\arcsin{\sqrt{\frac{m_1\Omega^2-d_1}{4c_1}}};\,\,\,k_2=\frac{2}{a}\arcsin{\sqrt{\frac{m_2\Omega^2-d_2}{4c_2}}};
$$
$$
g_1=\frac{a}{2\Omega}\sqrt{\left(\Omega^2-\frac{d_1}{m_1}\right)\left(\frac{4c_1+d_1}{m_1}-\Omega^2\right)};\,\,\,g_2=\frac{a}{2\Omega}\sqrt{\left(\Omega^2-\frac{d_2}{m_2}\right)\left(\frac{4c_2+d_2}{m_2}-\Omega^2\right)}.
$$

График зависимости коэффициента прохождения от жёсткости интерфейса $c_{12}$ построен в \href{https://github.com/mualal/waves-propagation/blob/master/mathematica/chain-chain-interface.nb}{блокноте Wolfram Mathematica}:

Далее \href{https://github.com/mualal/waves-propagation/blob/master/python/chain-chain-interface.ipynb}{в блокноте Jupyter} проведено сравнение численных результатов с аналитическим решением при $c_{12}\geqslant0$:



\end{document}
